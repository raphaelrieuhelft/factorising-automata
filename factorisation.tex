\documentclass[11pt, openany]{article}
\usepackage{pstricks,pstricks-add,pst-math,pst-xkey}
\usepackage[frenchb]{babel}
%\usepackage{slashbox}
\usepackage{graphicx}
\usepackage{amsmath,amssymb,amstext}
%\usepackage{comment}
\usepackage[utf8]{inputenc}
\usepackage[OT1]{fontenc}
\usepackage{pgf,tikz}
\usepackage{pgfplots}
\usepackage{floatpag}
\usepgfmodule{shapes}
\usetikzlibrary{arrows,patterns}
\usepackage[ps]{skak}
\usepackage{chessboard}
\newcounter{moncompteur}
\newtheorem{q}[moncompteur]{ \textbf{Question}}{}
\newtheorem{prop}[moncompteur]{ \textbf{Proposition}}{}
\newtheorem{df}[moncompteur]{ \textbf{Définition}}{}
\newtheorem{rem}[moncompteur]{ \textbf{Remarque}}{}
\newtheorem{theo}[moncompteur]{ \textbf{Théorème}}{}
\newtheorem{conj}[moncompteur]{ \textbf{Conjecture}}{}
\newtheorem{cor}[moncompteur]{ \textbf{Corollaire}}{}
\newtheorem{lm}[moncompteur]{ \textbf{Lemme}}{}
%\newtheorem{nota}[moncompteur]{ \textbf{Notation}}{}
%\newtheorem{conv}[moncompteur]{ \textbf{Convention}}{}
\newtheorem{exa}[moncompteur]{ \textbf{Exemple}}{}
\newtheorem{ex}[moncompteur]{ \textbf{Exercice}}{}
%\newtheorem{app}[moncompteur]{ \textbf{Application}}{}
%\newtheorem{prog}[moncompteur]{ \textbf{Algorithme}}{}
%\newtheorem{hyp}[moncompteur]{ \textbf{Hypothèse}}{}
\newenvironment{dem}{\noindent\textbf{Preuve}\\}{\flushright$\blacksquare$\\}
\newcommand{\cg }{[\kern-0.15em [}
\newcommand{\cd}{]\kern-0.15em]}
\newcommand{\R}{\mathbb{R}}
\newcommand{\K}{\mathbb{K}}
\newcommand{\N}{\mathbb{N}}
\newcommand{\Z}{\mathbb{Z}}
\newcommand{\C}{\mathbb{C}}
\newcommand{\U}{\mathbb{U}}
\newcommand{\Q}{\mathbb{Q}}
\newcommand{\B}{\mathbb{B}}
\newcommand{\card}{\mathrm{card}}
\newcommand{\norm}[1]{\left\lVert#1\right\rVert}
\pgfplotsset{compat=1.8}
\newcommand{\La}{\mathcal{L}}
\newcommand{\Ne}{\mathcal{N}}
\newcommand{\D}{\mathcal{D}}
\newcommand{\Ss}{\textsc{safestay}}
\newcommand{\Sg}{\textsc{safego}}
\newcommand{\M}{\textsc{move}}
\newcommand{\E}{\mathcal{E}}
\newcommand{\V}{\mathcal V}
\setlength{\parindent}{0pt}




\begin{document}
\floatpagestyle{plain}
\renewcommand{\labelitemi}{$\bullet$}

\title{Multiplication binaire et factorisation}
\date{}
\author{}
\maketitle



\section*{Position du problème}


On cherche un système dynamique discret qui, étant donné un entier non-premier, trouve un de ses diviseurs. On se base sur la multiplication binaire. En effet, on observe qu'une matrice à coefficients dans $\{0,1\}$ peut s'interpréter comme une multiplication binaire si ses lignes non-nulles sont égales entre elles, comme illustré dans la figure $1$. Les colonnes non-nulles sont alors aussi égales entre elles, et les deux facteurs de la multiplication sont alors lisibles, en base $2$, sur les lignes non-nulles pour l'un, et les colonnes non-nulles lues de bas en haut pour l'autre. Si on pose leur multiplication, toujours en base $2$, on retrouve en effet la matrice de départ, décalée par les retenues (cf. Figure $1$).



\begin{figure}[h]
\centering
\begin{minipage}[]{0.25\linewidth}

\begin{tabular}{cccc}
1&0&1&1\\
0&0&0&0\\
1&0&1&1\\
1&0&1&1\\
\end{tabular}

\end{minipage}
\quad
\begin{minipage}[]{0.4\linewidth}


\begin{tabular}{lllllllll|c}
&&&&&1&0&1&1&$\times$\\
\hline
&&&&&1&0&1&1&1\\
+&&&&0&0&0&0&.&0 \\
+&&&1&0&1&1&.&.&1\\
+&&1&0&1&1&.&.&.&1\\
\hline
&1&0&0&0&1&1&1&1&\\
\end{tabular}
\end{minipage}
\caption{La matrice de gauche s'interprète comme la multiplication de $11$ ($1011$ horizontalement) par $13$ ($1101$ verticalement, de haut en bas).}
\end{figure}

Par ailleurs, étant donné une matrice $M$ à coefficients dans $\{0,1\}$, on pose $\sigma(M)$ l'entier obtenu en sommant ses lignes décalées par des retenues (cf. Figure 2). Si $M$ représente une multiplication, $\sigma$ est par construction le produit. 

\begin{figure}[]
\centering
\begin{minipage}[]{0.3\linewidth}
$M=$
\begin{tabular}{cccc}
1&1&0&1\\
0&0&1&1\\
1&0&0&1\\
\end{tabular}
\end{minipage}
\quad
\begin{minipage}[]{0.6\linewidth}
$\sigma(M)=$
\begin{tabular}{llllllll}
&&&&1&1&0&1\\
+&&&0&0&1&1&.\\
+&&1&0&0&1&.&.\\
\hline
&&1&1&0&1&1&1\\
\end{tabular}
$=63$
\end{minipage}
\caption{Exemple de calcul de $\sigma$.}

\end{figure}

L'idée est donc de partir d'une matrice à coefficients dans $\{0,1\}$ pour laquelle $\sigma$ vaut le nombre qu'on cherche à factoriser, et de lui appliquer des transformations qui conservent $\sigma$ jusqu'à atteindre une matrice qui représente une multiplication.






\end{document}

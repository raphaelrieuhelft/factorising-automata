\documentclass[11pt, openany]{article}
\usepackage{pstricks,pstricks-add,pst-math,pst-xkey}
\usepackage[frenchb]{babel}
%\usepackage{slashbox}
\usepackage{graphicx}
\usepackage{amsmath,amssymb,amstext,amsthm}
%\usepackage{comment}
\usepackage[utf8]{inputenc}
\usepackage[OT1]{fontenc}
\usepackage{pgf,tikz}
\usepackage{pgfplots}
\usepackage{floatpag}
\usepgfmodule{shapes}
\usetikzlibrary{arrows,patterns}
\usepackage[ps]{skak}
\usepackage{chessboard}
\usepackage{import}
\newcounter{moncompteur}
\newtheorem{q}[moncompteur]{ \textbf{Question}}{}
\newtheorem{prop}[moncompteur]{ \textbf{Proposition}}{}
\newtheorem{df}[moncompteur]{ \textbf{Définition}}{}
\newtheorem*{df*}{ \textbf{Définition}}{}
\newtheorem{rem}[moncompteur]{ \textbf{Remarque}}{}
\newtheorem{theo}[moncompteur]{ \textbf{Théorème}}{}
\newtheorem{conj}[moncompteur]{ \textbf{Conjecture}}{}
\newtheorem{cor}[moncompteur]{ \textbf{Corollaire}}{}
\newtheorem{lm}[moncompteur]{ \textbf{Lemme}}{}
%\newtheorem{nota}[moncompteur]{ \textbf{Notation}}{}
%\newtheorem{conv}[moncompteur]{ \textbf{Convention}}{}
\newtheorem{exa}[moncompteur]{ \textbf{Exemple}}{}
\newtheorem{ex}[moncompteur]{ \textbf{Exercice}}{}
%\newtheorem{app}[moncompteur]{ \textbf{Application}}{}
%\newtheorem{prog}[moncompteur]{ \textbf{Algorithme}}{}
%\newtheorem{hyp}[moncompteur]{ \textbf{Hypothèse}}{}
\newenvironment{dem}{\noindent\textbf{Preuve}\\}{\flushright$\blacksquare$\\}
\newcommand{\cg }{[\kern-0.15em [}
\newcommand{\cd}{]\kern-0.15em]}
\newcommand{\R}{\mathbb{R}}
\newcommand{\K}{\mathbb{K}}
\newcommand{\N}{\mathbb{N}}
\newcommand{\Z}{\mathbb{Z}}
\newcommand{\C}{\mathbb{C}}
\newcommand{\U}{\mathbb{U}}
\newcommand{\Q}{\mathbb{Q}}
\newcommand{\B}{\mathbb{B}}
\newcommand{\card}{\mathrm{card}}
\newcommand{\norm}[1]{\left\lVert#1\right\rVert}
\pgfplotsset{compat=1.8}
\newcommand{\La}{\mathcal{L}}
\newcommand{\Ne}{\mathcal{N}}
\newcommand{\D}{\mathcal{D}}
\newcommand{\Ss}{\textsc{safestay}}
\newcommand{\Sg}{\textsc{safego}}
\newcommand{\M}{\textsc{move}}
\newcommand{\E}{\mathcal{E}}
\newcommand{\V}{\mathcal V}
\setlength{\parindent}{0pt}
\newcommand{\myrightleftarrows}[1]{\mathrel{\substack{\xrightarrow{#1} \\[-.6ex] \xleftarrow{#1}}}}
\newcommand{\longrightleftarrows}{\myrightleftarrows{\rule{1cm}{0cm}}}

\newcommand\tikzmark[1]{%
  \tikz[overlay,remember picture,baseline] 
  \node[anchor=base](#1){};}

\newcommand\MyLine[3][]{%
  \begin{tikzpicture}[overlay,remember picture]
    \draw[#1] (#2.north west) -- (#3.south east);
  \end{tikzpicture}}


\graphicspath{{.}}


\begin{document}
\floatpagestyle{plain}
\renewcommand{\labelitemi}{$\bullet$}

\title{Multiplication binaire et factorisation}
\date{}
\author{}
\maketitle



\section*{Position du problème}


On cherche un système dynamique discret qui, étant donné un entier non-premier, trouve un de ses diviseurs non-triviaux. On se base sur la multiplication binaire. En effet, on observe qu'une matrice à coefficients dans $\{0,1\}$ peut s'interpréter comme une multiplication binaire si ses lignes non-nulles sont égales entre elles, comme illustré dans la figure $1$. Les colonnes non-nulles sont alors aussi égales entre elles, et les deux facteurs de la multiplication sont alors lisibles, en base $2$, sur les lignes non-nulles pour l'un, et de bas en haut sur les colonnes non-nulles pour l'autre. Si on pose leur multiplication, toujours en base $2$, on retrouve en effet la matrice de départ décalée d'un cran par ligne vers la gauche (cf. Figure $1$).




\begin{figure}[h]
\centering
\begin{minipage}[]{0.25\linewidth}

\begin{tabular}{cccc}
1&0&1&1\\
0&0&0&0\\
1&0&1&1\\
1&0&1&1\\
\end{tabular}

\end{minipage}
\quad
\begin{minipage}[]{0.4\linewidth}


\begin{tabular}{lllllllll|c}
&&&&&1&0&1&1&$\times$\\
\hline
&&&&&1&0&1&1&1\\
+&&&&0&0&0&0&.&0 \\
+&&&1&0&1&1&.&.&1\\
+&&1&0&1&1&.&.&.&1\\
\hline
&1&0&0&0&1&1&1&1&\\
\end{tabular}
\end{minipage}
\caption{La matrice de gauche s'interprète comme la multiplication de $11$ ($1011$ horizontalement) par $13$ ($1101$ verticalement, de haut en bas).}
\end{figure}

Par ailleurs, étant donné une matrice $M$ à coefficients dans $\{0,1\}$, on pose $\sigma (M)$ l'entier obtenu en sommant ses lignes décalées (cf. Figure 2). Si $M$ représente une multiplication, $\sigma(M)$ est par construction le produit. 

\begin{figure}[]
\centering
\begin{minipage}[]{0.3\linewidth}
$M=$
\begin{tabular}{cccc}
1&1&0&1\\
0&0&1&1\\
1&0&0&1\\
\end{tabular}
\end{minipage}
\quad
\begin{minipage}[]{0.6\linewidth}
$\sigma(M)=$
\begin{tabular}{llllllll}
&&&&1&1&0&1\\
+&&&0&0&1&1&.\\
+&&1&0&0&1&.&.\\
\hline
&&1&1&0&1&1&1\\
\end{tabular}
$=55$
\end{minipage}
\caption{Exemple de calcul de $\sigma$.}

\end{figure}

L'idée est donc de partir d'une matrice $M$ à coefficients dans $\{0,1\}$ telle que $\sigma(M)$ vaut le nombre qu'on cherche à factoriser, et de lui appliquer des transformations qui préservent $\sigma$ comme invariant jusqu'à atteindre une matrice qui représente une multiplication. Ainsi, notre système n'a pas à vérifier que le produit de cette multiplication est l'entier à factoriser, ce qui simplifie considérablement sa conception.

\section*{Un algorithme}

Soit $n$ le nombre de chiffres en base $2$ du nombre composé $N$ dont on cherche un diviseur non-trivial. On se donne une matrice $M$ de taille $ n-1 \times \lceil \frac{n}{2}\rceil$ telle que $\sigma(M)=N$. Pour tous $a$ et $b$ différents de $1$ et $N$, et tels que $N = ab$, $a$ ou $b$ est de taille inférieure à $\lceil\frac{n}{2}\rceil$ et l'autre est de taille inférieure à $n-1$, on peut donc représenter leur multiplication à l'aide d'une matrice de la taille de $M$. On l'initialise en recopiant les $n-1$ bits de poids faible de $N$ sur la première ligne, et avec un $1$ en position $(2,1)$ et des zéros partout ailleurs (cf. Figure \ref{fig:init}). On vérifie aisément que $\sigma(M)=N$. 


\begin{figure}[h]
\centering

$21 \Rightarrow \overline{10101}^2 \Rightarrow$
\begin{tabular}{cccc}
0&1&0&1\\
1&0&0&0\\
0&0&0&0\\
\end{tabular}
$\Rightarrow$
\begin{tabular}{ccccccc}
&&&0&1&0&1\\
+&&1&0&0&0&.\\
+&0&0&0&0&.&.\\
\hline
&0&1&0&1&0&1\\
\end{tabular}

\caption{Exemple de construction de la matrice $M$ de départ pour factoriser $N=21$.}
\label{fig:init}
\end{figure}

Formellement, on pose $\La=\cg 1, n-1 \cd \times \cg 1, \lceil\frac{n}{2}\rceil\cd$ l'ensemble des cases de la matrice, et $\E=\{0,1\}^\La$ l'ensemble des configurations. Pour tout $x \in \E$, $(i,j)\in\La$, on note $x_{i,j}$ le coefficient $(i,j)$ de la matrice $x$. 

On se donne également un ensemble $\Gamma \subset \E^{\E\times\La}$ de transformations locales qui conservent $\sigma$. Elles sont représentées sur la figure 3, et s'appliquent aux coefficients indiqués en gras sous certaines conditions. Les règles \textbf{R5} et \textbf{R6} ne sont en fait pas nécessaires pour l'accessibilité des solutions : on les introduit par symétrie, pour ne pas privilégier de direction de déplacements des $1$. 

%On se donne donc un ensemble $\Gamma$ de six transformations locales qui conservent $\sigma$. Elles sont représentées sur la figure $3$. Plus 


\begin{figure}
\centering
\[
\begin{tabular}{cc}
\textbf 1& \\
 &0\\
\end{tabular}
\mathrel{\mathop{\longrightleftarrows}^{\mathrm{R1}}_{\mathrm{R2}}}
\begin{tabular}{cc}
\textbf 0& \\
 &1\\
\end{tabular}
\]


\[
\begin{tabular}{cc}
0& \\
\textbf 1&0\\
\end{tabular}
\mathrel{\mathop{\longrightleftarrows}^{\mathrm{R3}}_{\mathrm{R4}}}
\begin{tabular}{cc}
1& \\
\textbf 0&1\\
\end{tabular}
\]

\[
\begin{tabular}{ccc}
\textbf 1&0&\\
&&0\\
\end{tabular}
\mathrel{\mathop{\longrightleftarrows}^{\mathrm{R5}}_{\mathrm{R6}}}
\begin{tabular}{ccc}
\textbf 0&1&\\
&&1\\
\end{tabular}
\]

\caption{Transformations locales préservant $\sigma$.}
\end{figure}


%À chaque pas de temps, on choisit au hasard une case et une transformation de $\Gamma$, et on décide si on applique ou non la transformation.


Une formulation équivalente de l'égalité des lignes non-nulles est qu'un coefficient situé sur la même ligne qu'un 1 et sur la même colonne qu'un 1 vaut nécessairement 1 dans une matrice qui représente une multiplication. On va donc appliquer aléatoirement des règles conservant $\sigma$, en privilégiant les mouvements qui retirent les zéros de telles cases. Dans la suite, on dira qu'une case est \emph{sûre} s'il n'y a pas de $1$ sur la même ligne ou s'il n'y en a pas sur la même colonne, et \emph{dangereuse} s'il y a à la fois un $1$ sur la même ligne et un sur la même colonne.

Pour diriger les zéros vers les cases sûres, on voudrait n'effectuer que des mouvements qui sortent un zéro d'une case dangereuse, mais cette restriction est trop forte : il existe alors des situations bloquées, la figure $4$ en donne un exemple. Il est donc nécessaire d'autoriser au moins certains mouvements qui ne sont pas directement utiles en ce sens. Par ailleurs, on pourrait vouloir restreindre les mouvements des zéros vers des cases dangereuses, de la même façon que pour le problème des $n$ reines. Cependant, il se forme alors des blocs stables de $1$ : les zéros qui viennent remplacer un $1$ du bord du bloc en sont rapidement éjectés (les seules cases sûres proches sont à l'extérieur du bloc). 

On prend donc le parti de ne pas restreindre les mouvements des zéros vers des cases dangereuses, et on se donne une probabilité d'agitation $p$ : un mouvement valide mais qui ne sort pas un zéro d'une case dangereuse est effectué avec une probabilité $p$, alors qu'un mouvement valide qui sort un zéro d'une case dangereuse est toujours effectué. 



\begin{figure}
\centering
\begin{tabular}{cccccccc}
0&1&0&0&1&1&0&\fbox 0\\ 
0&\fbox 0&0&0&1&1&0&1\\
0&\fbox 0&0&0&1&1&0&1\\
0&0&0&0&0&0&0&0\\
0&\fbox 0&0&0&1&1&0&1\\
\end{tabular}


\caption{Exemple de situation bloquée si on interdit les mouvements ne déplaçant pas les zéros menacés (encadrés) : aucune des règles de la Figure $3$ ne s'applique à eux.}

\end{figure}

À chaque pas de temps, on choisit donc une case de $\La$ et une règle de $\Gamma$ au hasard, et on applique éventuellement la règle à la case. À titre d'exemple, si on choisit, dans une configuration $x$, la case $(i,j)$ et la règle \textbf{R1}, la nouvelle configuration est \[
\textbf{R1}(x,(i,j))=
\begin{cases}
  \hfill x' \hfill & \text{ si}  \begin{cases}\text{ $x_{i,j}=1$ et $x_{i+1,j+1}=0$ }\\
    \text{ et}  \left\{
        \begin{tabular}{@{}c@{}l}
          &\text{ $(i+1,j+1)$ est une case dangereuse}\\
          \text {ou}&\\
          &\text{ $\mathcal{B}(p)=1$}\\
        \end{tabular}
      \right.
  \end{cases}
  \\
  \hfill x \hfill & \text{ sinon}
\end{cases}
\]

avec $x'$ la configuration obtenue à partir de $x$ en échangeant les valeurs de $x_{i,j}$ et $x_{i+1,j+1}$, et $\mathcal{B}(p)$ une variable aléatoire valant $1$ avec une probabilité $p$ et $0$ avec une probabilité $1-p$.

On applique une règle si au moins un des zéros concernés est sur une case dangereuse : ainsi \[
\textbf{R3}(x,(i,j))=
\begin{cases}
  \hfill x'' \hfill & \text {si} \begin{cases}\text{ $x_{i,j}=1$ et $x_{i-1,j}=x_{i,j+1}=0$} \\
    \text{ et} \left\{
        \begin{tabular}{@{}c@{}l}
          &\text{ $(i-1,j)$ est une case dangereuse}\\
          \text {ou}& \text{ $(i,j+1)$ l'est}\\
          \text {ou}& \text{ $\mathcal{B}(p)=1$}\\
        \end{tabular}
      \right.
    \end{cases}
  \\
  \hfill x \hfill & \text {sinon}
\end{cases}
\]

avec $x''$ la configuration obtenue à partir de $x$ en changeant les valeurs de $x_{i,j}$, $x_{i-1,j}$ et $x_{i,j+1}$.

\medskip

Du fait de l'existence de mouvements spontanés, il n'y a pas de point fixe à proprement parler : on est contraint de vérifier après chaque mouvement si la matrice représente une multiplication valide, et d'arrêter les mises à jour le cas échéant. 

\subsection*{Atteignabilité de la solution}

On cherche ici à montrer que la solution est accessible à partir de la solution de départ en appliquant uniquement les règles de $\Gamma$. On va en fait montrer un résultat un peu plus général : en appliquant les règles de $\Gamma$, il est possible de passer d'une matrice quelconque à n'importe quelle autre matrice pour laquelle $\sigma$ a la même valeur. 

\begin{df*}
Pour toutes matrices $A$ et $B$, on a $A\sim B$ si et seulement si on peut passer de $M$ à $N$ en appliquant des transformations de $\Gamma$.
\end{df*}

Remarquons que $\sim$ est une relation d'équivalence (la symétrie vient du fait que toutes les règles de $\Gamma$ ont un inverse, donc tous les mouvements sont réversibles).

Pour toute matrice $M$, on numérote ses diagonales descendantes en commençant à $0$ par celle réduite au coin supérieur droit (cf Fig. \ref{fig:diags}), et on note $b_i$ le nombre de coefficients égaux à $1$ sur la diagonale $i$, de sorte que $\sigma (M) = \sum\limits_ib_i2^i$. On constate que deux matrices ayant les mêmes coefficients $b_i$ pour tout $i$ sont équivalentes pour $\sim$ (en fait, on peut passer de l'une à l'autre en utilisant uniquement les règles \textbf{R1} et \textbf{R2}). On note de plus $r$ le nombre de diagonales descendantes de la matrice (par exemple, dans le cas d'une matrice $3\times 3$, $r=5$). Enfin, pour tout $0\leq i < r$, on note $t_i$ la taille de la $i$-ème diagonale : on a $0\leq b_i \leq t_i$. 
 


\begin{figure}[hbt]
\centering
%\begin{tikzpicture}
%\
%\begin{matrix}
%\tikzmark{a} & \tikzmark{b} & \tikzmark{c} & \\
%\phantom{0}\tikzmark{a21}& & \phantom{0}\tikzmark{a23}& b_0 \\
%\phantom{0}\tikzmark{a31}& \phantom{0}\tikzmark{a32} &  \phantom{0}\tikzmark{a33} & b_1 \\
%&b_4&b_3&b_2\\
%\end{matrix}
%\]
%\end{tikzpicture}

\scalebox{0.3}{\subimport{.}{figdiag.pdf_t}}

\caption{Signification des coefficients $b_i$.}
\label{fig:diags}
\end{figure}

On va montrer que pour tout $N$, toute matrice pour laquelle $\sigma=N$ est équivalente pour $\sim$ à une certaine matrice dont le nombre de coefficients égaux à $1$ est minimal.  

Soit donc $N \in \N$, $m$ le nombre minimal de $1$ que doit contenir une matrice de la taille voulue pour laquelle $\sigma=N$. Soit $M$ une telle matrice. Elle vérifie nécessairement, pour tout $i<r-1$, $b_i \leq 1$ ou $b_{i+1}=t_{i+1}$ : sinon, on pourrait appliquer des règles \textbf{R1} et \textbf{R2} dans les diagonales $i$ et $i+1$, puis une fois la règle \textbf{R4} et se ramener à une matrice avec strictement moins de $1$.
Par conséquent, si $b_i > 1$, on a par récurrence que pour tout $j>i$, $b_j = t_j$.
On suppose que tous les coefficients de $M$ ne sont pas égaux à $1$ (sinon, $M$ est la seule matrice pour laquelle $\sigma = N$ et on a fini).
Il existe alors $k \in \cg 0, r-1 \cd$ tel que $b_k<t_k$, pour tout  $i<k$, $b_i \leq 1$ et pour tout $i>k$, $b_i=t_i$ (c'est le plus grand $i$ tel que $b_i < t_i$).

Soit maintenant $M'$ une autre matrice telle que $\sigma (M')=N$, et minimale pour le nombre de coefficients égaux à $1$ dans sa classe d'équivalence pour $\sim$. Notons, pour tout $i$, $b'_i$ le nombre de coefficients égaux à $1$ sur la $i$-ème diagonale de $M'$. Il existe également $k'$ tel que $b'_{k'} < t_{k'}$, pour tout $i<k'$, $b'_i \leq 1$ et pour tout $i>k'$, $b'_i=t_i$. Si $k'<k$, alors pour tout $i>k$, $b_i= b'_i$ donc : 

%\begin{align*}
%something\\
%something else
%\end{align*}

\begin{align*}
\sum\limits_{i=0}^{r-1}b_i 2^i = N &= \sum\limits_{i=0}^{r-1}b'_i 2^i\\
\sum\limits_{i=0}^k b_i 2^i &= \sum\limits_{i=0}^k b'_i 2^i\\
b_k 2^k + \underbrace{\sum\limits_{i=0}^{k-1} b_i 2^i}_{\text{$< 2^k$ car $b_i \leq 1$ pour $i<k$}} &=\underbrace{\sum\limits_{i=k'+1}^k t_i 2^i + b_{k'} 2^{k'} + \sum\limits_{i=0}^{k' -1} b_i 2^i}_{\geq t_k 2^k}\\
\end{align*}
contradiction, car $b_k<t_k$. D'où $k \leq k'$, et par symétrie $k = k'$. Mais alors pour tout $i<k$, $b_i = b'_i$ par unicité de la décomposition en base $2$ du reste de $N$ modulo $2^k$, et pour tout $i>k$, $b_i = b'_i = t_i$, donc on a aussi $b_k = b'_k$ (car $\sum\limits_{i=0}^{r-1} b_i 2^i = \sum\limits_{i=0}^{r-1} b'_i 2^i = N$).

D'où $M' \sim M$. On a donc montré que pour toute matrice $M"$ telle que $\sigma(M")=N$, si $M'$ est la matrice avec le moins de $1$ dans la classe d'équivalence de $M"$, $M"\sim M' \sim M$. D'où par transitivité, toute matrice pour laquelle $\sigma=N$ est équivalente à $M$, donc toutes les matrices pour lesquelles $\sigma=N$ sont équivalentes entre elles.



\section*{Performances et observations}

Les performances varient grandement selon l'entier à factoriser : sa dé\-composition en facteurs premiers, et plus généralement la distribution de $1$ dans la ou les solutions semblent être des facteurs déterminants. Ainsi, les nombres les plus faciles semblent être ceux qui ont le plus de diviseurs, ce qui s'explique par le fait que plus de solutions existent pour de tels nombres. Si on ne s'intéresse qu'aux nombres qui sont le produit de deux nombres premiers, les nombres pour lesquelles les matrices solutions du problème comportent peu de $1$ semblent les plus difficiles : l'algorithme trouve en $10^6$ pas de temps en moyenne un diviseur de $493$ ($ = 17\times 29$), mais pour factoriser $362$ ($ = 2\times 181$), il lui faut $10^9$ étapes ! Ceci est dû au fait que la densité de $1$ varie très peu au cours du temps (elle oscille autour d'une densité moyenne, et les solutions pour lesquelles la densité finale en est éloignée sont difficiles à trouver). 

\medskip

De plus, les conditions que l'on s'est donné pour appliquer les règles tendent à chasser les zéros menacés, mais rien ne pousse particulièrement les $1$ isolés à se déplacer. Il est donc relativement difficile de trouver les positions pour lesquelles les $1$ sont tous à un endroit précis de la matrice : ainsi, pour les nombres de la forme $2^kp$ avec $p$ premier, la matrice aura tous ses $1$ sur la même ligne (la $k$-ième) ou bien sur la même diagonale. Ainsi, dans les cas ``faciles'' tels que $493$, une solution est atteinte environ $40$ fois plus vite avec $p=0.05$ qu'avec $p=1$ (cas où le système applique des règles valides sans tenir compte du caractère dangereux ou sûr des cases), mais dans les cas difficiles tels que $362$, les performances sont plutôt meilleures avec la marche aléatoire $p=1$. Plus généralement, l'évolution du système n'est dirigée que faiblement, notamment parce qu'on a décidé de ne pas restreindre les déplacements de zéros vers des cases dangereuses et parce que le système n'a pas de mémoire. 

\medskip
 
Le fait que les $1$ isolés ne soient pas encouragés à se déplacer est d'autant plus gênant qu'on a choisi une taille de matrice nécessairement trop grande pour les facteurs que l'on va trouver. En effet, on a pris une matrice de taille $n-1 \times \lceil \frac{n}{2} \rceil$ pour un nombre de taille $n$ afin d'assurer de pouvoir trouver n'importe quel diviseur non trivial, mais la somme des longueurs des deux facteurs que l'on va trouver est au plus $n+1$. Par conséquent, tous les $1$ d'une matrice solution sont contenus dans un rectangle dont l'un des sommets est le coin supérieur droit et dont le périmètre est limité : s'il y a un $1$ dans le coin inférieur droit, il ne peut y en avoir aucun dans toute la moitié gauche de la matrice ! Si, dans une exécution de l'algorithme, aucun $1$ n'apparaît typiquement dans le quart inférieur gauche pour $p$ assez petit, il y en a fréquemment à la fois dans le coin supérieur gauche et inférieur droit, ce qui fait perdre beaucoup de temps. 

%TODO : comparatif entre différents entiers


\section*{Quelques améliorations}

\subsection*{Fixer la taille des diviseurs recherchés}

On peut donc envisager l'amélioration suivante : plutôt que d'utiliser une grande matrice, on suppose connue la taille des diviseurs. Si $n$ est la taille du nombre à factoriser, il y a au plus $2n-1$ tailles de matrices possibles à essayer car la somme des tailles des diviseurs vaut nécessairement $n$ ou $n+1$ et on ignore les matrices dont une dimension vaut $1$ car on cherche des diviseurs non-triviaux. On lance donc en parallèle l'algorithme sur les $2n-1$ matrices différentes, et on arrête l'exécution dès qu'une solution est trouvée pour l'une des tailles. Quitte à supposer impair le nombre à factoriser, on impose de plus qu'il y ait un $1$ à chacun des quatre coins (c'est une condition nécessaire pour qu'une matrice dont les dimensions sont exactement les tailles de deux facteurs soit solution : chaque facteur commence par un $1$ car les zéros à gauche ne sont pas significatifs, et finissent par un $1$ car les diviseurs d'un nombre impair sont impairs). 

\medskip
Il apparaît que l'exécution de l'algorithme est significativement plus rapide qu'avec la grande matrice. Les nombres dont l'un des facteurs est très petit deviennent très faciles à traiter, car la matrice de la bonne taille est très petite, et le calcul y aboutit donc rapidement. Les nombres les plus difficiles à traiter deviennent, comme pour les algorithmes classiques de factorisation, les produits de deux nombres premiers de même taille : en effet, la seule matrice pour laquelle l'algorithme peut aboutir est alors de taille maximale. Même dans ce cas, l'exécution est plus rapide car on ne perd plus de temps à cause des $1$ situés dans une zone inutile de la matrice. On gagne ainsi un facteur $100$ sur le temps de factorisation de $329$ ($ =7\times 47$), largement de quoi justifier le lancement simultané d'une quinzaine d'exécutions. % TODO : stats plus détaillées ici ?

%TODO : comparatif entre différents entiers
%TODO : comparatif entre fact0 et fact_fixed_lengths

\subsection*{Permettre des échanges à plus longue distance}

Les transformations préservant $\sigma$ qu'on a introduites plus haut sont très localisées : on peut envisager de remplacer les règles \textbf{R1} et \textbf{R2} par une règle plus générale, qui consiste à intervertir deux cellules situées sur la même diagonale descendante. De même, on peut remplacer les règles \textbf{R3} et \textbf{R5} par une règle qui consiste à échanger un $1$ d'une diagonale contre deux sur la diagonale à sa droite, et les règles \textbf{R4} et \textbf{R6} par une règle qui consiste à échanger un $0$ d'une diagonale contre deux sur la diagonale à sa droite. Les transformations du système sont ainsi plus équitables au sens où on peut maintenant atteindre, en appliquant une seule règle, des positions qu'on ne pouvait précédemment atteindre que par une séquence précise de transformations locales. 

Il semble cependant que ce changement n'améliore pas significativement le temps mis pour atteindre une solution : pour certains entiers à factoriser, on constate une amélioration de l'ordre de $10\%$, mais pour d'autres, le temps de convergence subit au contraire une augmentation du même ordre.   

%TODO : comparatif entre fact_fixed_lengths et fact_global

\subsection*{N'utiliser que des informations locales}

Dans l'algorithme ci-dessus, on utilise des informations globales sur le système lorsqu'on décide si on applique une transformation : en effet, on décide si une case est sûre ou dangereuse en comptant les $1$ sur la ligne et la diagonale concernées. On peut en fait ajuster l'algorithme pour se restreindre à utiliser uniquement des informations locales, en mettant en place un système de signaux analogue à celui utilisé pour le problème des $n$ reines. 

On procède de la façon suivante : les cellules de la matrice, en plus de contenir un $0$ ou un $1$, peuvent contenir des signaux en provenance des quatre directions $W$, $E$, $NW$ et $SE$. On souhaite qu'un signal en provenance d'une certaine direction soit présent sur une cellule s'il existe un coefficient égal à $1$ dans cette direction. Ainsi, une case est dite \emph{sûre} si elle ne contient pas de signal en provenance du nord-ouest ni du sud-est, ou bien si elle ne contient pas de signal en provenance de l'ouest ni de l'est. Dans le cas contraire, elle est dite \emph{dangereuse}. On a ainsi traduit l'ancienne définition d'une cellule sûre (pas de $1$ sur la même diagonale, ou bien pas de $1$ sur la même ligne) en terme de signaux.

À chaque pas de temps, on choisit une cellule $c$ à mettre à jour et on commence par mettre à jour ses signaux de la même façon que pour le problème des $n$ reines : pour chaque direction $d$, si $c'$ est la cellule voisine de $c$ dans la direction $d$, le signal en $c$ en provenance de la direction $d$ devient présent si $c'$ contient un $1$, et prend le même état (présent ou absent) que celui de $c'$ en direction de $d$ sinon.
Après avoir mis les signaux de $c$ à jour, on applique l'algorithme précédent pour décider si on applique une transformation de $\Gamma$, en utilisant la nouvelle définition de case sûre/dangereuse. 

Comme on pouvait s'y attendre, on constate une déterioration des performances lorsqu'on se restreint ainsi à des informations locales. Cependant, cette augmentation du temps mis à atteindre une solution reste raisonnable, ce qui est prometteur dans le sens où se restreindre à des informations locales peut permettre de paralléliser le système en effectuant plusieurs mises à jour simultanées, sans conflits à condition que les zones affectées par les différentes mises à jour soient disjointes. Ceci était moins efficace avec l'algorithme initial du fait de la nécessité de garder une trace du nombre total de $1$ sur chaque ligne et diagonale, ce qui impose de centraliser les mises à jour sous peine d'en effectuer certaines avec des informations erronées.

%TODO : comparatif entre fact_fixed_lengths et fact_sigs

\nocite{Pom96}
\nocite{BahiC06}
\nocite{Fat13}

\bibliographystyle{plain}
\bibliography{bibfact}

\end{document}

